%! Author = dch
%! Date = 2/20/24

% Preamble
\documentclass[11pt]{article}

% Packages
\usepackage{amsmath}
\usepackage{tikz}
\usepackage{pgfplots}
\usepackage{tkz-fct}
\usepackage{pst-plot}
\pgfplotsset{compat=newest} % This ensures that we use the latest configuration and features of pgfplots

% Document
\begin{document}

\section*{Exercise List for Lab 0}

\subsection*{Wolfram Alpha in General}
\begin{enumerate}
    \item Find the value of the thousandth digit in the decimal expansion of the number $\pi$.
    \item Provide the current distance from the Moon to the Earth.
    \item Give the relative frequency of occurrence of letters in the alphabet in Polish text.
    \item Check the weather on your birthday in your city.
    \item Check how much vitamin D is in 1 km$^3$ of milk.
    \item Compare Poland and Germany.
    \item Compare an apple and an orange.
    \item Compare Albert Einstein and Marie Curie.
    \item Check the ratio of the population of France to Germany.
    \item Check how many calories are in 10 M\&Ms and separately in 0.5 l of vodka.
    \item Check the distance between your city and Turin.
    \item Check the time simultaneously in Poland and Chile.
    \item Check the effect of entering each of the phrases: NATO, Sierpiński Triangle, 5000 words in Polish.
\end{enumerate}

\subsection*{Math with Wolfram Alpha}
\begin{enumerate}
    \item Simplify the quotient $(x^3-1)/(x-1)$.
    \item Plot the function $\sin(x^2)/x$.
    \item Draw the graph of the function $\sin(x^2)/x$ covering the interval from 10 to 20.
    \item Check how to express $\sin(2\alpha)$ as a function of $\sin(\alpha)$ and $\cos(\alpha)$.
    \item Calculate the sum of the reciprocals of successive natural numbers from 1 to 10000.
    \item Calculate the sum of the reciprocals of the squares of all natural numbers.
    \item Factorize the number 1234567890.
    \item Expand the expression $(x+1)(x-2)$.
    \item Find the factored form of the expression $2 - 5x - 3x^2$.
    \item Determine in how many ways 6 different numbers can be chosen from 49.
    \item How many permutations are there of a 15-element set?
    \item Draw the set of solutions of the equation $x^2 + y^2 = 1$.
    \item Draw the set of solutions of the equation $x^2 + y^3 = 1$.
    \item Find all asymptotes of the function $f(x) = \frac{x^2 -1}{x^2-4}$.
    \item Find all asymptotes of the function $f(x) = \frac{x^2 -1}{x-2}$.
    \item Solve the equation $\sin(x) = \cos(x)$.
    \item Solve the equation $\sin(x) = \cos(2x)$.
    \item Solve the equation $\cos(x) = x/\pi$.
    \item Draw 3D graph of the function $f(x,y)=\sin(\sqrt{x^2+y^2})/\sqrt{x^2+y^2}$.
    \item Plot $\sin(1x)$, $\sin(2x)$, $\sin(3x)$, $\sin(4x)$ on the same plot.
\end{enumerate}

\subsection*{Physics with Wolfram Alpha}
\begin{enumerate}
    \item What color corresponds to a 480 nm wave?
    \item What gravitational acceleration do we have on planets in the Solar system?
    \item 10 nearest stars.
    \item Spring pendulum $l_0=0.12m$, $l_i=0.24m$, $\theta_i=80^\circ$.
    \item Joule's law $u=3V$, $R=1\Omega$ for 10s.
    \item Add velocities, $200000$ km/s, $200000$ km/s.
    \item Add velocities, $0.9c$, $0.9c$.
    \item Atomic spectrum of nitrogen.
    \item Calculate the diameter of a silicon atom in nanometers.
    \item Single slit diffraction $d=1/100$ inch, $\lambda=500nm$.
    \item RLC circuit $10\Omega$, $12H$, $400\mu F$.
    \item Photon energy $435nm$.
    \item Spring pendulum $l_0=0.12m$, $l_i=0.24m$, $\theta_i=80^\circ$.
    \item Orbital path of Hubble telescope.
    \item Find distance between volcano Vesuvius and Warsaw and establish time when we will hear eruption in Warsaw.
\end{enumerate}

\newpage

\section*{Problems for GPT}

\subsection*{Problem 1}

Compute a derivative of the following function:

\begin{itemize}
    \item $f(x) = x^2 + 3x - 5$
    \item $x(t) = t^2 + 3t - 5$
\end{itemize}

What is a difference between these two derivatives?

\subsection*{Problem 2}

Compute an integral of the following function:

\begin{itemize}
    \item $f(x) = x^2 + 3x - 5$
    \item $x(t) = t^2 + 3t - 5$
\end{itemize}

What is a difference between these two integrals?

\subsection*{Problem 3}

Plot the following parametric function:

\begin{align*}
 x(t) &= 3t \\
 y(t) &= t^2 - 3t
\end{align*}

What is the shape of the plot?
What describes the parametric function?
Can you tell what kind of physical process is described by the function?

\end{document}
